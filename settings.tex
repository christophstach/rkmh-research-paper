% \documentclass[conference]{IEEEtran}
\documentclass[conference,compsoc]{IEEEtran}
% \documentclass[journal]{IEEEtran}
% \documentclass[10pt,journal,compsoc]{IEEEtran}
% \documentclass[journal,comsoc]{IEEEtran}
% \documentclass[journal,transmag]{IEEEtran}

\usepackage[utf8]{inputenc}
\usepackage[T1]{fontenc}
% \usepackage[ngerman]{babel}
\usepackage{ifpdf}
\usepackage[numbers]{natbib}

\ifCLASSINFOpdf
  \usepackage[pdftex]{graphicx}
  \graphicspath{{../pdf/}{../jpeg/}}
  \DeclareGraphicsExtensions{.pdf,.jpeg,.png}
\else
  \usepackage[dvips]{graphicx}
  \graphicspath{{../eps/}}
  \DeclareGraphicsExtensions{.eps}
\fi

\usepackage{amsmath}
\usepackage{algorithmic}
\usepackage{array}

% IEEEtran contains the IEEEeqnarray family of commands that can be used to
% generate multiline equations as well as matrices, tables, etc., of high
% quality.

\ifCLASSOPTIONcompsoc
  \usepackage[caption=false,font=footnotesize,labelfont=sf,textfont=sf]{subfig}
\else
  \usepackage[caption=false,font=footnotesize]{subfig}
\fi

\usepackage{stfloats}
\usepackage{url}