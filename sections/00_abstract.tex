\begin{abstract}
    The rkmh toolkit is a collection of algorithms and filtering methods recently released. It is used for the classification of reads of DNA sequencing technologies. The authors of the algorithm state, that it is especially useful and more accurate than than its predecessors mash and sourmash in finding co-infections of different but similar sublineages of viruses. They demonstrate their results for the Human papillomavirus (HPV). While presenting the result their paper lacks explaining underlying technologies. This work aims to supplement the original paper of the rkmh by explaining different techniques sued by the rkmh toolkit. Also the author compares it to other algorithm and gives a brief perspective about alternative use cases.
\end{abstract}
  