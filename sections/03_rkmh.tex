\newpage

\section{Rkmh}

The rkmh toolkit uses the MinHash algorithm to convert reads (documents in context of text mining) to sketches (signatures in context of text mining). It first uses MurmurHash3 \cite{murmurHash}, a non-cryptographic hash function, to convert k-mers (k-shingles / k-grams in the context of text mining) to 64bit integer values. Before generating the sketch-signature, rkmh uses different kind of filters to the reduce the number of hashes the sketch is calculated of.\\

Rkmh offers different filters. It can remove hashes that total number of occurrences doesn't meet a defined threshold. Seldom occurring hashes are prone to be sequencing errors. Also k-mers that occur to many times can be filtered, because they are shared by many DNA sequences and are not informative for classifying.\\

More filters are available and are described in the original paper \cite{rkmh}. By providing the different filters the number of hashes the sketch is calculated of can be reduced by a great amount. This has positive impact on the performance of the toolkit.\\

Rkmh enriches each read it analyzes with a possibility to one or more reference genomes. By doing that it can filter reads that do not clearly contribute to a references. This helps removing reads which are uncertain in identifying a single references. With filtered and MinHashed reads, rkmh is able to classify large datasets of of DNA seqeunces and find infections and co-infections of the Human Paplimomus Virus (HPV) or other viral genomes. Rkms is written in C++ with OpenMP \cite{openMP} support for parallelisation and works extremly fast. The authors state that classification can be done within minutes on a standard laptop.