\subsection{Hash functions}
A hash function $ h $ is a usually quickly to compute function which takes  items of any set $ S $ and maps it to a fixed size number $ n $. Therefore it generates a signature for the items. The signatures upper bound is fixed by $ n $. Having enough items in $ S $, it can happen that different items are assigned the same signature and are not distinguishable any more. This is called a collision.\\

\begin{table}[h!]
    \centering
    \begin{tabular}{| c | c | c | c | c | c |}
        \hline
        0  & 1  & 2  & 3  & 4  & 5  \\
        \hline
           & 30 &    & 10 &    & 11 \\
        \hline
           &    &    &    &    & 41 \\
        \hline
    \end{tabular}   
\end{table}
    
In the example above we assign signatures to the items of the set $ S = \{ 10,11,30,41 \} $. As hash a function serves $ h(x) = (2x+1) \mod 6 $.  In this case $ n $ is $ 6 $ and the numbers $ 11 $ and $ 41 $ get the same signature assigned, which leads to a collision.\\

Hash functions exist for different purposes. There are many used in the field cryptographics \cite{cryptographicHashFunctions}, which try to avoid collisions. But also non-cryptographic hash function are available, which do not share this property. The second kind of hash function is used in dimensionality reduction and similarity estimation of datasets \cite{practicalHashFunctions}.\\