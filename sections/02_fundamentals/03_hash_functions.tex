\subsection{Hash functions}
A hash function $ h $ is a usually quickly to compute function which takes  elements of any set $ S $ and maps it to a fixed size number $ n $, the so called hash. The hash's upper bound is fixed by $ n $. Having enough elements in $ S $, it can happen that the same hash is assigned different elements. This leads to a so called collision and elements are not distinguishable anymore.\\

In the example the hash function $ h(x) = (2x+1) \mod 6 $ assigns hashes to the elements of set $ S = \{ 10,11,30,41 \} $.  In this case $ n $ is $ 6 $ and the numbers $ 11 $ and $ 41 $ get the same hash assigned, which leads to a collision.\\

\begin{table}[H]
    \centering
    \begin{tabular}{| c | c | c | c | c | c | c |}
        \hline
        \textbf{Hash} & \textbf{0} & \textbf{1} & \textbf{2} & \textbf{3} & \textbf{4} & \textbf{5}  \\
        \hline
        \multirow{2}{*}{\textbf{Elements}}   & 30 &    & 10 &    & 11 & \\
        &    &    &    &    & 41  &\\
        \hline
    \end{tabular}    
    \caption{Hash table of set $ S $. Each element of $ S $ has a different  hash with some colliding with each other.}
\end{table}

Hash functions exist for different purposes. User passwords of websites get often stored as hash instead of writing the clear text password to the database \cite{cryptographicHashFunctions}. They are also used to identify documents on a file system or in the field of text mining for dimensionality reduction \cite{practicalHashFunctions}.\\

In the next section the author describes how to use hash functions to skip the generation of permutations of row indices, which are used by the MinHash algorithm.\\

\subsubsection{Overcoming the permutation problem of MinHashing with hash functions} Instead of using different permutations of row indices it is also possible to use different hash functions. Hashing all the shingles of set $ S $ to numerical values and sorting them afterwards, will organize the shingles in a pseudorandom order without story the permutation in memory. The only caveat is that some shingles will produce a hash collision. Though this will hardly carry any weight on generating the end result \cite{minhash}.\\

Another solution is to hash the shingles with a appropriate hash function to a fixed size number $ n $. This creates collisions but reduces also the space the permutations take in memory. The author of the original publication \cite{minhash}, recommend using Rabin fingerprints \cite{rabinFingerprinting}, which spread the hashed elements equally and very densely over the available space of $ n $.