\subsection{MinHash}
\label{ssec:minhash}

MinHashing is a often used technique to compare big datasets. In the field text mining \citeauthor{minhash} first applied it to compare a large corpus of crawled web pages \cite{minhash}. It works by reducing the dimensionality of sequences (in the context of web pages often called documents). The algorithm assigns a signature to each document. The goal is that similar documents share similar signatures. Comparing two signatures approximates the Jaccard coefficient and can be done in a fraction of time than calculating the actual jaccard coefficient. Also the field bio informatics deals wich large numbers of sequences. Making them quickly comparable is therefore highly beneficial. \\
To generate the signature of a document, it first needs to be stored in a more compact form.

\subsubsection{Matrix representation}

Let the union of the shingled documents $ S_k(A) $ and $ S_k(B) $ represent the rows of a matrix and the documents the column. An element of the matrix is $ 1 $ if the shingle occurs in the document otherwise the element is $ 0 $. This typically leads to a sparse matrix, where most of the elements in the matrix are $ 0 $. Though in this example the matrix is not sparse. The data from the previous example represents the matrix $ M $. Also a matrix $ P $ of random permutations of row indices needs to be generated.\\

\begin{equation}
    \begin{split}
        P = 
        \begin{pmatrix}
            0 & 3 & 5 & 4 & 4 \\
            1 & 6 & 4 & 1 & 0 \\
            2 & 2 & 1 & 2 & 3 \\
            3 & 5 & 3 & 6 & 1 \\
            4 & 1 & 2 & 3 & 6 \\
            5 & 0 & 0 & 5 & 2 \\
            6 & 4 & 6 & 0 & 5
        \end{pmatrix}
        M = 
        \begin{pmatrix}
            1 & 0 \\
            1 & 1 \\
            1 & 1 \\
            1 & 1 \\
            1 & 0 \\
            0 & 1 \\
            0 & 1
        \end{pmatrix}   
    \end{split}
\end{equation} \\

The permutations serves in the next step to calculate the signature for the columns in the next step.\\

\subsubsection{Generation of a signature}

A signature represents a document. In the field of bio informatics, literature talks about sketches representing sequences. This is effectively the same, described with different words in different contexts.\\

With the permutations $ P $ generated before, it is possible to calculate a signature for each column.\\

To generate the signature matrix $ S $ of the columns, $ M $ is ordered by the first random permutations in $ P $. For each column in $ M $, the first row index in which the element is $ M $ is taken for the new signature matrix $ S $. This creates the matrix $ S $ with the same number of columns like $ M $ and the same number of rows like the number of columns of $ P $. Each column now represents a MinHash value for a document.\\

\begin{equation}
    S = 
    \begin{pmatrix}
        0 & 1 \\
        1 & 0 \\
        1 & 1 \\
        1 & 1 \\
        1 & 1
    \end{pmatrix}
\end{equation} \\

Counting the overlapping components of the signatures matrix and dividing it by the total number of components results in a value wich estimates the Jaccard coefficient. For this example the estimated Jaccard coefficient is $ \frac{2}{5} = 0.4 $ which is fairly close to the real one with a value of $ \frac{3}{7} \approx 0.43 $.\\

The problem with this approach is that generating and storing the permutations $ P $ can take a lot of space in memory for large datasets. Also sorting the complete matrix for each permutations is a time consuming task. These problems can be overcome by using hash functions instead of permutations. The next section describes how hash functions work. Afterwards the author explains they can be used in the process of permutation generation.\\

