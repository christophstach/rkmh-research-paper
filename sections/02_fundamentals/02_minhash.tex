\subsection{MinHash}
\label{ssec:minhash}

MinHashing is a often used technique to compare big datasets. I is used in data-mining and was first applied by \citeauthor{minhash} to compare a large corpus of crawled web pages \cite{minhash}. It works by reducing the dimensionality of sequences (also often called documents). The algorithm assigns a signature to a document. These signature signatures have the property to reflect the similarity of the documents. \\

\subsubsection{Shingling}

Given two documents $ A = "abccdab" $ and $ B = "bccdacb" $. The documents are first converted into a sets of shingles with the size of $ k $. For $ k = 3 $ the resulting sets are: \\

\begin{equation}
    \begin{split}
        S_k(A) = \{abc, bcc, ccd, cda, dab\} \\
        S_k(B) = \{bcc, ccd, cda, dac, acb\} \\
        S_k(A) \cup S_k(B) = \{abc, bcc, ccd, cda, dab, dac, acb\}
    \end{split}
\end{equation}\\

Documents and shingles can now be represented as a sparse matrix, where the shingles are the rows and the documents the columns. The computed Jaccard coeeficcient of A and B is  $ J(S_k(A),S_k(B)) = \frac{3}{7} \approx 0.43 $.\\

\subsubsection{Matrix representation}

For the purpose of this algorithm we build a sparse matrix of all shingles of the union of $ S_k(A) $ and $ S_k(B) $. Instead of writing the shingles directly to the matrix, we write a $ 1 $ if the shingle occurs in the document otherwise we write a $ 0 $. With the data from the previuose this leads to a the matrix representation $ M $. The columns represent a document and the rows the shingles. \\

\begin{equation}
    \begin{split}
        P = 
        \begin{pmatrix}
            0 & 3 & 5 & 4 \\
            1 & 6 & 4 & 1 \\
            2 & 2 & 1 & 2 \\
            3 & 5 & 3 & 6 \\
            4 & 1 & 2 & 3 \\
            5 & 0 & 0 & 5 \\
            6 & 4 & 6 & 0
        \end{pmatrix}
        M = 
        \begin{pmatrix}
            1 & 0 \\
            1 & 1 \\
            1 & 1 \\
            1 & 1 \\
            1 & 0 \\
            0 & 1 \\
            0 & 1
        \end{pmatrix}   
    \end{split}
\end{equation} \\

Also we store a matrix $ P $ of random permutations of the row indecies. We use them to generate the signature for the columns on the next step\\  



\subsubsection{Generation of a signature}





% Hier weitermachen