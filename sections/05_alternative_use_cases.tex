\section{Alternative use cases}

MinHashing as a lot of application in text mining. It can be used to make large collection of documents searchable and comparable. The MinHash is calculated for each document and then compared to the MinHash of search query.\\

Another possible application is a toolkit for checking for plagiarism of scientific documents. Given a large database of documents, the MinHash is calculated and stored for each document. After calculating the MinHash for a newly submitted documents, it can be compared to all documents in the database. A similarity threshold determines if the document is a plagiarism or not. Also similar documents can be shown for a manual review. Filtering options, like in rkmh, can reduce the total number of shingles and improve the results and performance of the toolkit.\\

Also music songs can be compared with MinHashing. Music songs are sequences of notes. Therefore they can be shingled. The application would work similarly to the plagiarism toolkit or rkmh. A MinHash for each song is calculated and stored along with a song in a database. An application could record the sounds in the direct environment of a microphone and compare it to the songs stored in the database to find a song the user is currently listening to.\\